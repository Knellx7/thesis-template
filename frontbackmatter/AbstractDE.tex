%*******************************************************
% Abstract in German
%*******************************************************
\begin{otherlanguage}{ngerman}
	\pdfbookmark[1]{Zusammenfassung}{Zusammenfassung}
	\chapter*{Zusammenfassung}
	Viele Unternehmen haben in den letzten Jahren Erfahrungen mit Docker gesammelt. Mit zunehmender Komplexität der Anwendungen wird zum weiteren Ausbau der Container Infrastruktur oft Kubernetes als Lösung für die Container Orchestration\footnote{\label{foot:orchestration}Die Organisation von Containern} verwendet. \medskip
	
	Mit der Verwendung einer solchen Lösung und dem damit einhergehenden Wechseln in ein oft schnelllebiges dynamisches Umfeld stellt die persistente Speicherung von Daten eine Herausforderung dar.  \medskip
    
    Wie auch bei der Verwendung von Containern sind Daten grundsätzlich nur solange vorhanden wie die Anwendung existiert. Um dies zu verhindern, bietet Docker und auch Kubernetes eine Vielzahl an Möglichkeiten, die Daten persistent zu speichern.  \medskip
    
    Die Bachelorthesis gibt einen Überblick über verschiedene Persistenzlösungen in Kubernetes und die Technologie hinter ihnen. Basierend auf dieser Analyse werden die Lösungen nach einem Anforderungskatalog evaluiert und eine Handlungsempfehlung für ein existierendes Kubernetes Cluster abgeleitet. Anschließend werden die Erkenntnisse auf dieses Cluster angewandt und die Persistenzlösung eingerichtet.
\end{otherlanguage}

% Evtl noch kurz Ergebniss anteasern